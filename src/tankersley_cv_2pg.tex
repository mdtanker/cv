\documentclass{ExpressiveResume}

% ----- Resume -----
\begin{document}


% ----- Name + Contact Information -----
\resumeheader[
    firstname=Matthew,
    middleinitial=D,
    lastname=Tankersley,
    email=matt.d.tankersley@gmail.com,
    % phone=123-456-7890,
    linkedin=matthew-tankersley,
    github=mdtanker,
    orcid=0000-0003-4266-8554,
    % city=Wellington,
    % state=NZ,
    qrcode=./images/qr.png,
    fixobjectivespacing=true
]

\objective{A recent Ph.D. graduate specialized in airborne geophysical
    analysis and inversion \\
    applied to the cryosphere, with a dedication to conducting open-source and reproducible science.
}


% ----- Education -----
\section{Education}
\experience{Ph.D.}{Geophysics}{September 2019}{October 2023}{
    Airborne geophysical investigation beneath Antarctica’s Ross Ice Shelf \newline
    Antarctic Research Center, Victoria University of Wellington, New Zealand \newline
}

\experience{Bachelor of Arts (with distinction)}{Geology}{August 2014}{May 2018}{
    distinction in Geology (GPA 3.7) and a minor in Physics (GPA 3.9) \newline
    Thesis: "Aerogeophysical analysis of crustal structures under the
    Ross Ice Shelf" \newline
    Colorado College, Colorado Springs, USA \newline
}


% ----- Publications -----
\section{Publications}
\begin{cventries}
    \paper{2022}
    {Basement topography and sediment thickness beneath Antarctica’s Ross Ice Shelf}
    {Geophysical Research Letters}
    {Matthew Tankersley, Huw Horgan, Christine Siddoway, Fabio Caratori Tontini, and Kirsty Tinto.}
    {10.1029/2021GL097371}
    {5 citations}

    \paper{2019}
    {Ross Ice Shelf response to climate driven by the tectonic imprint on seafloor bathymetry}
    {Nature Geoscience}
    {Kirsty Tinto, Laurence Padman, Christine Siddoway, Scott Springer, ... Matthew Tankersley}
    {10.1038/s41561-019-0370-2}
    {104 citations}
\end{cventries}


% ----- Presentations -----
% \vspace{-2em}
\section{Selected Presentations}
\noindent ORAL PRESENTATIONS \newline
\begin{cventries}
    \talk{2023}
    {Progress towards an open-source geometric gravity inversion with stochastic uncertainty estimates}
    {AGU Fall Meeting, San Francisco, CA, USA}
    {}
    {}
    {}

    \talk{}
    {Addressing bathymetry uncertainty beneath the Ross Ice Shelf}
    {New Zealand-Australia Antarctic Science Conference, Christchurch, NZ}
    {Slides: \url{https://doi.org/10.6084/m9.figshare.24412021.v1}}
    {}
    {}

    \talk{2021}
    {Sediment thickness and basement depths beneath the Ross Ice Shelf from aeromagnetic data}
    {New Zealand Antarctic Science Conference, Christchurch, NZ}
    % {}
    {}
    {}

\end{cventries}

\vspace{-1em}
\noindent POSTER PRESENTATIONS \newline
\begin{cventries}

    \talk{2023}
    {Gravity inversion as a method to recover sub-ice shelf bathymetry; applied to the Ross Ice Shelf}
    {Scientific Committee on Antarctic Research, Instabilities \&
        Thresholds in Antarctica, Trieste, Italy}
    {Poster: \url{https://doi.org/10.6084/m9.figshare.24117420.v2}}
    {}
    {}

    \talk{}
    {Antarctic-Plots: a Python package to help conduct Antarctic research}
    {\newline 1) Scientific Committee on Antarctic Research, Instabilities \&
        Thresholds in Antarctica, Trieste, Italy
        \newline \hspace*{1em} \textbf{Awarded best poster}
    \newline 2) New Zealand-Australia Antarctic Science Conference, Christchurch, NZ
        \newline \hspace*{1em} \textbf{Awarded 2nd best poster}
    \newline 3) AGU Fall Meeting, San Francisco, CA, USA
        }
    {Posters: \url{https://doi.org/10.6084/m9.figshare.21183931}}
    {}

    \talk{2022}
    {Revealing sub-ice shelf sediment basins with airborne magnetics}
    {West Antarctic Ice Sheet Conference and Workshop, Estes Park, CO, USA}
    {Poster: \url{https://doi.org/10.6084/m9.figshare.21172042.v2}}
    {}
    {}

    \talk{} %2022
    {Antarctic-Plots: a Python package to help download, visualize, and present Antarctic datasets}
    {\newline 1) West Antarctic Ice Sheet Conference and Workshop, Estes Park, CO, USA
        \newline 2) The Future of Geodetic-Geophysical Observational Networks in
        Antarctica Workshop (SCAR-INSTANT), Fort Collins, CO, USA
    }
    {Poster: \url{https://doi.org/10.6084/m9.figshare.21183931.v3}}
    {}
    {}

    % \talk{2021}
    % {New contribution to Ross Ice Shelf (Antarctica) boundary conditions: basement depths and sediment thickness determined from aeromagnetic data}
    % {AGU, virtual participation, presented by Christine Siddoway}
    % {Abstract: \url{https://agu.confex.com/agu/fm21/meetingapp.cgi/Paper/988486}}
    % {}
    % {}

    % \talk{2020}
    % {Broad basement structures under Antarctica’s Ross Ice Shelf revealed from aeromagnetic data}
    % {AGU, virtual participation}
    % {Abstract: \url{https://agu.confex.com/agu/fm20/meetingapp.cgi/Paper/714573}}
    % {}
    % {}

    % \talk{} %2020
    % {Constrained geopotential modelling of the ocean cavity and geology beneath the Ross Ice Shelf}
    % {Geoscience Societ of New Zealand annual conference, Christchurch, NZ}
    % {Abstract: \url{https://agu.confex.com/agu/fm20/meetingapp.cgi/Paper/714573}}
    % {}
    % {}

    % \talk{2018}
    % {Aerogeophysical analysis of crustal structures under the Ross Ice Shelf}
    % {AGU, Washington D.C., USA}
    % {Abstract: \url{https://agu.confex.com/agu/fm18/meetingapp.cgi/Paper/442287}}
    % {}
    % {}
\end{cventries}


% ----- Software -----
\vspace{-3em}
\section{Open-source Software Development}
\begin{description}
    \item[Invert4Geom:] 3D geometric gravity inversions (\url{https://invert4geom.readthedocs.io/})
        \item Founder and core-maintainer
    \item[Antarctic-Plots:] Functions to automate Antarctic data visualization (\url{https://antarctic-plots.readthedocs.io/})
        \item Founder and core-maintainer
    \item[Fatiando a Terra:] Open source tools for geophysics (\url{https://www.fatiando.org})
        \item Contributor
\end{description}

% ----- Skills -----
\vspace{1em}
\section{Technical Skills}
\begin{description}
    \item[Programming] Python, GMT
    \item[Python packages] Pandas, Xarray, NumPy, SciPy, Dask, PyGMT,
        Matplotlib, Plotly, Pooch, Verde, Harmonica, Optuna, GeoPandas, Shapely
    \item[Markup] Markdown, \LaTeX, Curvenote
    \item[OS] Linux, Windows
    \item[Other tools] Geosoft Oasis Montaj, Jupyter Notebooks, git,
        GitHub, VS Code, Binder,
        ReadTheDocs, QGIS, LibreOffice Suite, Microsoft Office Suite
\end{description}
\vspace{3em}


% ----- Field work -----
\section{Field work}
\experience{Antarctica - Kamb Ice Stream}{Geophysical field assistant}{November 2019}{December 2019}{
    \achievement{
        Worked within a team of 5 stationed in a remote field camp on the
        Ross Ice Shelf conducting an \tech{active source seismic survey} and
        a \tech{gravity survey}.
    }
    \achievement{
        Included training and extensive use of snowmobiles, Hagglund tracked
        vehicles, transport, wiring, and detonation of explosive charges,
        operation of a hot water drill for emplacing charges at a 20m depth,
        and deploying a 1 km array of geophones.

    }
    \achievement{
        Other duties included \tech{planning and executing the gravity
            survey}, GNSS surveying the gravity and seismic stations, and
        setting up and maintaining camp infrastructure.
    }
}

\experience{Antarctica - Discovery Deep}{Geophysical field assistant}{December 2021}{Febuary 2022}{
    \achievement{
        Similar to above but in a field camp consisting of just our team of
        5. Additional survey methods included seismic surveying with a
        streamer of geophones and surface detonation of det-cord.
    }
    \achievement{
        Shared all duties of our self-contained camp (cooking, cleaning,
        camp safety etc.).
    }
}

\experience{RV Tangaroa - TAN2006}{Marine Seismic Assistant}{July 2020}{August 2020}{
    \achievement{
        Worked aboard the RV Tangaroa conducting a \tech{marine seismic} and
        \tech{multibeam bathymetry} survey of the Chatham Rise, New Zealand.
    }
    \achievement{
        Duties included monitoring seismic data collection and
        pre-processing of multibeam bathymetry data.
    }
}

\experience{Western USA}{Geologic Fieldwork}{2014}{2018}{
    \achievement{
        Over 100 days of geologic fieldwork throughout the Western USA
        during my undergraduate degree.
        This included geologic and structural mapping, stratigraphic
        profiles, and soil and rock sample collection.
    }
}


\end{document}