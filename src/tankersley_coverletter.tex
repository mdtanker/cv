\documentclass{ExpressiveCoverLetter}

\begin{document}

\coverletterheader[
    firstname=Matthew,
    middleinitial=D,
    lastname=Tankersley,
    email=matt.d.tankersley@gmail.com,
    % phone=123-456-7890,
    linkedin=matthew-tankersley,
    github=mdtanker,
    % orcid=0000-0003-4266-8554,
    % city=Wellington,
    % state=New Zealand
]
\vspace{-0.1in}
% \vspace{0.15in}
% \today
% \vspace{0.10in}


Dear Innovation Award Subcommittee,

I am writing to express my interest in applying for the Innovation Award
for Cryospheric Sciences.
Throughout my Ph.D. studying Antarctica's Ross Ice Shelf, I found there
were parts of my research workflow that were inefficient, consuming
valuable time. These tasks included manually downloading and archiving
various Antarctic datasets, performing common geospatial analysis
techniques, and creating figures.
To help me conduct my research more efficiently, I decided to write
Python code to accomplish these tasks.
I found this Python code significantly increased my research productivity, reproducibility, and quality.
To share these benefits with other Antarctic researchers, I published
this code as an open-source Python package called
\textit{Antarctic-Plots}.
It is hosted on GitHub
(\url{https://github.com/mdtanker/antarctic_plots}) and has extensive
documentation and tutorials (\url{https://antarctic-plots.readthedocs.io/en/latest/}).

The \textit{Antarctic-Plots} package contains five core modules.
\textit{Fetch} handles the download, storage, and retrieval of many
commonly used Antarctic datasets, eliminating the need to
remember file paths or re-download datasets. \textit{Regions} has
pre-defined geographic boundaries of commonly studied areas, enabling easy investigation into a
specific region without having to remember coordinates. \textit{Maps}
and \textit{Profile} provide functions for creating publication-quality
figures. \textit{Utils} contains a collection of functions, such as
filtering, resampling, or detrending geospatial data.

I have presented this Python package at several conferences, receiving
two poster awards and many thanks from fellow researchers who use the
software.
Antarctic-Plots has been a
side project that I have developed in my personal time. While the package
is already publically available, there is much more work needed to add
thorough testing to the code, add more datasets and features, and
promote the package at conferences and workshops.

Funding through the Innovation Award will help me to spend more time
developing this tool which I believe will be well-used by a wide
community of Antarctic researchers.

Sincerely,

% \vspace{.2in}

Matt Tankersley

\end{document}