\documentclass{ExpressiveCoverLetter}

\begin{document}

\coverletterheader[
    firstname=Matthew,
    middleinitial=D,
    lastname=Tankersley,
    email=matt.d.tankersley@gmail.com,
    % phone=123-456-7890,
    linkedin=matthew-tankersley,
    github=mdtanker,
    % city=Wellington,
    % state=New Zealand
]

\vspace{0.25in}
\today
\vspace{0.15in}


Dear Dr. Heagy,

I am writing to express interest in the recently advertised postdoctoral
position "Computational geophysics for mineral exploration and mining" at the
UBC Geophysical Inversion Facility.
A little bit about me; I am from
Colorado, USA, and received my undergraduate degree in Geology from
Colorado College in 2018.
I then moved to Wellington, New Zealand, in
2019 to begin my Ph.D. at the Antarctic Research Centre at the Victoria
University of Wellington.
I submitted my Ph.D. thesis two weeks ago and will defend in the coming months.

During this Ph.D., I used airborne gravity and magnetic data to
investigate the Earth beneath the floating Ross Ice Shelf in Antarctica.
My primary objectives were to better understand the components of the
Earth which influence how the ice sheet behaves. These components
include the geologic composition and shape of
the seafloor beneath the ice shelf. The first chapter of the thesis used
airborne magnetics data collected over the entire ice shelf and a
depth-to-magnetic source technique, Werner deconvolution, to model the contact
between magnetic basement rock, and overlying sediments. From this, we
found a continuous drape of non-magnetic sediments over the seafloor, as
well as several distinct depocenters with several kilometers of
sediment. This research was published open-access in Geophysical Research
Letters (\url{https://doi.org/10.1029/2021GL097371}) and the code and
Jupyter notebooks are available in a Github repository (\url{https://github.com/mdtanker/RIS_basement_sediment}).


The second main objective of my Ph.D. was to model the bathymetry
beneath the Ross Ice Shelf through a gravity inversion. This style
of non-linear geometric gravity inversion for bathymetry is commonly
used in sub-ice shelf settings but is almost exclusively accomplished with
proprietary, expensive, and poorly documented inversion algorithms (e.g.
Oasis Montaj). Instead, I choose to develop a new inversion written in Python, utilizing several existing
tools, such as \textit{Harmonica} for forward gravity calculations and
\textit{Scipy} for least-squares regression.
% As a means of testing the inversion's capabilities, and learning about
% the sensitivity of the inversion to various data components, such as
% noise in the gravity data and the number of \textit{a priori}
% bathymetry measurements, I performed a suite of inversions on
% synthetic datasets.
I first tested this inversion on a suite of synthetic models and finally
used the inversion to create a new model of bathymetry beneath the Ross
Ice Shelf, revealing some large differences with past models of
bathymetry.
% From this work, I plan in the next few months to publish a methods paper on the inversion and a results paper on the Ross Ice Shelf.
If you would like any more details, I am happy to share my submitted
thesis with you.

I am passionate about geophysical exploration and its applications to
studying the Earth. While I haven't worked directly with EM data, I
would be very excited to incorporate a new tool into my repertoire. I
believe my knowledge of inversion theory, my experience coding in
Python, both independently and in a collaborative setting (my
contributions to Fatiando packages), and my commitment to conducting open-source
research makes me well-suited for this postdoc. Additionally, I think my
past experiences and collaborations could bring interesting
opportunities for using EM data to study the sub-ice environment of Antarctica.


Thank you for taking the time to review my application. I look forward
to hopefully meeting with you to further discuss my
qualifications for a this postdoc. If you have any questions, please feel free to
email me at \href{mailto:matt.d.tankersley@gmail.com}{matt.d.tankersley@gmail.com}.


For letters of reference, please feel free to use the below contacts:
\begin{itemize}
    \item Huw Horgan, Ph.D. Primary Supervisor: \href{mailto:huw.horgan@slf.ch}{huw.horgan@slf.ch}
    \item Fabio Caratori-Tontini, Ph.D. Secondary Supervisor: \href{mailto:fabio.caratori.tontini@unige.it}{fabio.caratori.tontini@unige.it}
    \item Christine Siddoway, Collaborator: \href{mailto:csiddoway@coloradocollege.edu}{csiddoway@coloradocollege.edu}
\end{itemize}


Sincerely,

\vspace{.15in}

Matt Tankersley

\end{document}