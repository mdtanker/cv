\documentclass{ExpressiveResume}

% ----- Resume -----
\begin{document}


% ----- Name + Contact Information -----
\resumeheader[
    firstname=Matthew,
    middleinitial=D,
    lastname=Tankersley,
    email=matt.d.tankersley@gmail.com,
    % phone=123-456-7890,
    linkedin=matthew-tankersley,
    github=mdtanker,
    orcid=0000-0003-4266-8554,
    % city=Wellington,
    % state=NZ,
    qrcode=./images/qr.png,
    fixobjectivespacing=true
]

\objective{A recent Ph.D. graduate specialized in airborne geophysical
    analysis and inversion \\
    applied to the cryosphere, with a dedication to conducting open-source and reproducible science.
}


% ----- Education -----
\section{Education}
\experience{Ph.D. (submitted)}{Geophysics}{September 2019}{June 2023}{
    Airborne Geophysical Investigation beneath Antarctica’s Ross Ice Shelf \newline
    Antarctic Research Center, Victoria University of Wellington, New Zealand \newline
}

\experience{Bachelor of Arts (with distinction)}{Geology}{August 2014}{May 2018}{
    distinction in Geology (GPA 3.7) and a minor in Physics (GPA 3.9) \newline
    Thesis: "Aerogeophysical Analysis of Crustal Structures under the
    Ross Ice Shelf" \newline
    Colorado College, Colorado Springs, USA \newline
}


% ----- Work Experience -----
\section{Work Experience}

\experience{Teaching assistant}{Victoria University of
    Wellington}{Febuary 2021}{June 2022}{
    \achievement{
        Developed coursework for and led laboratory and fieldwork portions of 3rd-year undergraduate courses \tech{Applied
            Geophysics} and \tech{Field Geophysics}.
    }
}

\experience{Paraprofessional of Geology}{Colorado College}{August 2018}{June 2019}{
    \achievement{
        Planned, led and evaluated laboratory portions of
        undergraduate geology courses.
    }
    \achievement{
        Organized logistics for up to week-long field trips for 20+ students.
    }
}

\experience{Summer Intern}{Lamont-Doherty Earth Observatory}{June 2017}{August 2017}{
    \achievement{
        Analyzed correlations between properties of
        Greenland glacial earthquakes (magnitudes and force azimuths) with seasonality and calving front
        positions; utilizing \tech{Python}, \tech{Generic Mapping Tools}, and
        \tech{Landsat imagery}.
    }
    \achievement{
        Worked with Dr. Kira Olsen and Dr. Meredith Nettles.
    }
}

\experience{Summer Intern}{USGS}{June 2016}{August 2016}{
    \achievement{
        Collected ground-based \tech{gravity and magnetic} data and conducted
        geologic mapping to aid in a geothermal play fairway analysis of
        the Pacific Northwest of the USA.
    }
    \achievement{
        In conjunction with Colorado College, the USGS, and Washington State
        DNR.
    }
}

% \experience{White water raft guide}{Three Rivers Resort, Colorado}{June 2015}{August 2015}{
% }
% \experience{Waiter}{Three Rivers Smokehouse, Colorado}{June 2015}{August 2015}{
% }


% ----- Awards and Honors -----
\section{Awards and Honors}
\begin{multicols}{2}
    \begin{itemize}
        \item Best poster at SCAR-INSTANT 2023 Conference \hfill 2023
        \item 2nd best poster at NZ-Australia Antarctic Science
              Conference \hfill 2023
        \item SCAR-INSTANT ECR travel grant \hfill 2023
        \item NZ-Australia Antarctic Science Conference student travel grant \hfill 2023
        \item Arnold Heine Antarctic Research Award \hfill 2023
        \item Endowed Development Fund \hfill 2022
        \item NZ Antarctic Science Conference travel grant \hfill 2021
        \item Antarctica NZ Doctoral Scholarship \hfill 2020-2022
        \item Antarctic Science Platform PhD Scholarship \hfill 2020-2023
        \item Estwing Outstanding Senior Geologist Award \hfill 2018
        \item William A. Fischer Family Scholarship \hfill 2018
        \item Witter Family Fund \hfill 2017
        \item Patricia Buster Research Scholarship Fund \hfill2016
        \item Dean's list, Colorado College (4 semesters) \hfill 2014-2018

    \end{itemize}
\end{multicols}


% ----- Publications -----
\section{Publications}
\noindent PEER-REVIEWED SCIENTIFIC ARTICLES \newline
\begin{cventries}
    \paper{2022}
    {Basement Topography and Sediment Thickness Beneath Antarctica’s Ross Ice Shelf}
    {Geophysical Research Letters}
    {Matthew Tankersley, Huw Horgan, Christine Siddoway, Fabio Caratori Tontini, and Kirsty Tinto.}
    {10.1029/2021GL097371}
    {5 citations}

    \paper{2019}
    {Ross Ice Shelf response to climate driven by the tectonic
        imprint on seafloor bathymetry}
    {Nature Geoscience}
    {Kirsty Tinto, Laurence Padman, Christine Siddoway, Scott Springer, ... Matthew Tankersley}
    {10.1038/s41561-019-0370-2}
    {104 citations}
\end{cventries}


\noindent IN-PREP SCIENTIFIC ARTICLES \newline
\begin{cventries}
    \inpreppaper{2023}
    {Gravity inversion: a tool for bathymetry modelling}
    {Matthew Tankersley, Huw Horgan, and Fabio Caratori Tontini.}

    \inpreppaper{2023}
    {Bathymetry depths and uncertainties beneath Antarctica's Ross Ice Shelf}
    {Matthew Tankersley, Huw Horgan, and Fabio Caratori Tontini.}

\end{cventries}


% ----- Presentations -----
\section{Presentations}
\noindent ORAL PRESENTATIONS \newline
\begin{cventries}
    \talk{2023}
    {Addressing bathymetry uncertainty beneath the Ross Ice Shelf}
    {New Zealand-Australia Antarctic Science Conference, Christchurch, NZ}
    {Slides: \url{https://doi.org/10.6084/m9.figshare.24412021.v1}}
    {}
    {}

    \talk{2021}
    {Sediment thickness and basement depths beneath the Ross Ice Shelf from aeromagnetic data}
    {New Zealand Antarctic Science Conference, Christchurch, NZ}
    % {}
    {}
    {}

\end{cventries}

\noindent POSTER PRESENTATIONS \newline
\begin{cventries}

    \talk{2023}
    {Gravity inversion as a method to recover sub-ice shelf bathymetry; applied to the Ross Ice Shelf}
    {Scientific Committee on Antarctic Research, Instabilities \&
        Thresholds in Antarctica, Trieste, Italy}
    {Poster: \url{https://doi.org/10.6084/m9.figshare.24117420.v2}}
    {}
    {}

    \talk{}
    {Antarctic-Plots: a Python package to help conduct Antarctic research}
    {\newline 1) Scientific Committee on Antarctic Research, Instabilities \&
        Thresholds in Antarctica, Trieste, Italy,
        \textbf{Awarded best poster}
        \newline 2) New Zealand-Australia Antarctic Science Conference, Christchurch, NZ,
        \textbf{Awarded 2nd best poster}}
    {Poster: \url{https://doi.org/10.6084/m9.figshare.21183931.v3}}
    {}
    {}

    % \talk{}
    % {Antarctic-Plots: a Python package to help conduct Antarctic research}
    % {New Zealand-Australia Antarctic Science Conference, Christchurch, NZ \&
    % \textbf{Awarded 2nd best poster}}
    % % {Poster: \url{}}
    % {}
    % {}


    \talk{2022}
    {Revealing sub-ice shelf sediment basins with airborne magnetics}
    {West Antarctic Ice Sheet Conference and Workshop, Estes Park, CO, USA}
    {Poster: \url{https://doi.org/10.6084/m9.figshare.21172042.v2}}
    {}
    {}

    \talk{} %2022
    {Antarctic-Plots: A Python package to help download, visualize, and present Antarctic datasets}
    {\newline 1) West Antarctic Ice Sheet Conference and Workshop, Estes Park, CO, USA
        \newline 2) The Future of Geodetic-Geophysical Observational Networks in
        Antarctica Workshop (SCAR-INSTANT), Fort Collins, CO, USA
    }
    {Poster: \url{https://doi.org/10.6084/m9.figshare.21183931.v3}}
    {}
    {}

    \talk{2021}
    {New Contribution to Ross Ice Shelf (Antarctica) Boundary Conditions: Basement Depths and Sediment Thickness Determined from Aeromagnetic Data}
    {AGU, virtual participation, presented by Christine Siddoway}
    {Abstract: \url{https://agu.confex.com/agu/fm21/meetingapp.cgi/Paper/988486}}
    {}
    {}

    \talk{2020}
    {Broad basement structures under Antarctica’s Ross Ice Shelf revealed from aeromagnetic data}
    {AGU, virtual participation}
    {Abstract: \url{https://agu.confex.com/agu/fm20/meetingapp.cgi/Paper/714573}}
    {}
    {}

    \talk{} %2020
    {Constrained geopotential modelling of the ocean cavity and geology beneath the Ross Ice Shelf}
    {Geoscience Societ of New Zealand annual conference, Christchurch, NZ}
    {Abstract: \url{https://agu.confex.com/agu/fm20/meetingapp.cgi/Paper/714573}}
    {}
    {}

    \talk{2018}
    {Aerogeophysical analysis of crustal structures under the Ross Ice Shelf}
    {AGU, Washington D.C., USA}
    {Abstract: \url{https://agu.confex.com/agu/fm18/meetingapp.cgi/Paper/442287}}
    {}
    {}

\end{cventries}


\section{Field work}
\experience{Antarctica - Kamb Ice Stream}{Geophysical field assistant}{November 2019}{December 2019}{
    \achievement{
        Worked within a team of 5 stationed in a remote field camp on the
        Ross Ice Shelf conducting an \tech{active source seismic survey} and
        a \tech{gravity survey}.
    }
    \achievement{
        Included training and extensive use of snowmobiles, Hagglund tracked
        vehicles, transport, wiring, and detonation of explosive charges,
        operation of a hot water drill for emplacing charges at a 20m depth,
        and deploying a 1 km array of geophones.

    }
    \achievement{
        Other duties included \tech{planning and executing the gravity
            survey}, GNSS surveying the gravity and seismic stations, and
        setting up and maintaining camp infrastructure.
    }
}

\experience{Antarctica - Discovery Deep}{Geophysical field assistant}{December 2021}{Febuary 2022}{
    \achievement{
        Similar to above but in a field camp consisting of just our team of
        5. Additional survey methods included seismic surveying with a
        streamer of geophones and surface detonation of det-cord.
    }
    \achievement{
        Shared all duties of our self-contained camp (cooking, cleaning,
        camp safety etc.).
    }
}

\experience{RV Tangaroa - TAN2006}{Marine Seismic Assistant}{July 2020}{August 2020}{
    \achievement{
        Worked aboard the RV Tangaroa conducting a \tech{marine seismic} and
        \tech{multibeam bathymetry} survey of the Chatham Rise, New Zealand.
    }
    \achievement{
        Duties included monitoring seismic data collection and
        pre-processing of multibeam bathymetry data.
    }
}

\experience{Western USA}{Geologic Fieldwork}{2014}{2018}{
    \achievement{
        Over 100 days of geologic fieldwork throughout the Western USA
        during my undergraduate degree.
        This included geologic and structural mapping, stratigraphic
        profiles, and soil and rock sample collection.
    }
}

\section{Open-source Software Development}
\begin{cventries}
    \multiline{Since 2022}
    {Fatiando a Terra: Open source tools for geophysics}
    {Contributor \newline \url{https://www.fatiando.org}}

    \multiline{Since 2022}
    {Antarctic-Plots: Functions to automate Antarctic data visualization}
    {Founder and core-maintainer \newline \url{https://antarctic-plots.readthedocs.io/en/latest/}}
\end{cventries}


\section{Technical Skills}
\begin{description}
    \item[Programming] Python, GMT
    \item[Python packages] Pandas, Xarray, NumPy, SciPy, Dask, PyGMT,
        Matplotlib, Plotly, Pooch, Verde, Harmonica, Optuna, GeoPandas, Shapely
    \item[Markup] Markdown, \LaTeX, Curvenote
    \item[OS] Linux, Windows
    \item[Other tools] Geosoft Oasis Montaj, Jupyter Notebooks, git,
        GitHub, VS Code, Binder,
        ReadTheDocs, QGIS, LibreOffice Suite, Microsoft Office Suite
\end{description}


\section{Reviewer}
\begin{itemize}[label={}, leftmargin=*]
    \item \href{https://www.tandfonline.com/toc/tnzg20/current}{New
              Zealand Journal of Geology and Geophysics}
\end{itemize}


\section{Qualifications}
\begin{description}
    \item[First Aid Level 1] St John, Wellington, Nov 2022
    \item[Backcountry Avalanche Course] New Zealand Snow Safety Institute,
        Sep 2021
    \item[Basic Snowcraft Course] New Zealand Alpine Club, Aug 2021
    \item[Wilderness First Aid Course] National Outdoor Leadership
        School, 2019
    \item[AIARE 1 Avalanche Safety Course] Colorado College, 2018
\end{description}


\section{Recreational Interests}
\begin{description}
    \item[Outdoor recreation] Mountaineering, backcountry skiing, rock climbing, hiking, mountain biking
    \item[International travel] Having lived in 6 countries across 5
        continents, I have a keen interest in international
        cultures and easily adapt to new locations.
\end{description}

\end{document}